\documentclass[letterpaper,aps,pra,10pt,floats,nofootinbib,notitlepage]{revtex4-1}
\usepackage[latin1]{inputenc}
\usepackage[T1]{fontenc}
\usepackage{subfigure}
\usepackage{amsmath,amssymb,mathbbol}

\usepackage{grffile,graphicx,color}
\usepackage{hyperref}

\newcommand{\w}{\omega}
\newcommand{\Int}{\int\limits}
\newcommand{\lf}{\texttt{laserfields}}
\newcommand{\plotlf}{\texttt{plotlaserfourier}}
\newcommand{\printlf}{\texttt{printlaserfourier}}
\DeclareMathOperator{\erf}{erf}
\newcommand{\titlestr}{Units of Fourier-transformed fields in \lf{}}

\hypersetup{colorlinks,%
hypertexnames=true,%
pdfauthor={Johannes Feist},%
pdftitle={\titlestr},%
linkcolor=blue,%
citecolor=blue,%
urlcolor=blue}

\begin{document}

\title{\titlestr}
\author{Johannes~Feist}
\email{johannes.feist@gmail.com}
\affiliation{Departamento de F�sica Te�rica de la Materia Condensada, Universidad Aut�noma de Madrid}
\date{\today}

\maketitle

We discuss the units of the Fourier transform of the laser fields as calculated in the \lf{} library and the included \plotlf{} and \printlf{} programs.

The instantaneous intensity of an electric field is $I(t)=\frac{\epsilon_0 c}{2} E^2(t)$, which gives
$I(t) = 3.50944\cdot10^{16}\,E_{au}^2(t)\ $W/m$^2$ if the electric field is given in atomic units, $E(t)=E_{au}(t) \mathcal{E}_{au}$ 
(where $\mathcal{E}_{au}=\frac{e}{4\pi\epsilon_0 a_0^2} = 5.142207\cdot10^{11}\ $V/m is the atomic unit of electric field strength).
In atomic units, we have $I(t) = 5.45249\,E^2_{au}(t)\ $a.u., where ``a.u.'' here is Hartree/$(t_{au} a_0^2)$, with Hartree$\ =27.211\ $eV,
atomic unit of time $t_{au}=24.188\ $as, and Bohr radius $a_0= 5.29177\cdot10^{-11}\ $m.


The Fourier transform as calculated in \lf{} is
\begin{equation}
  \tilde E(\w) = \frac{1}{\sqrt{2\pi}} \Int_{-\infty}^\infty e^{-i \w t} E(t) \mathrm{d}t\,,
\end{equation}
with all quantities in atomic units. This means that $\tilde E(\w)$ from \lf{} is in atomic units, which are
$\mathcal{E}_{au} t_{au}=1.24384\cdot10^{-5}\ $V\,s/m. According to the
\href{http://en.wikipedia.org/wiki/Fourier_transform}{Plancherel theorem}, we have
\begin{equation}
\Int_{-\infty}^{\infty} E^2(t) \mathrm{d}t = \Int_{-\infty}^{\infty} |\tilde E(\w)|^2 \mathrm{d}\w \,
\end{equation}
where we have used that $E(t)$ is real, i.e., $|E(t)|^2=E^2(t)$.
We use this to define a ``spectral intensity'' $\tilde I(\w)$
\begin{equation}
\rho_{tot} = \Int_{-\infty}^{\infty} I(t) \mathrm{d}t = \frac{\epsilon_0 c}{2} \Int_{-\infty}^{\infty} E^2(t) \mathrm{d}t = \frac{\epsilon_0 c}{2} \Int_{-\infty}^{\infty} |\tilde E(\w)|^2 \mathrm{d}\w = \Int_{-\infty}^{\infty} \tilde I(\w) \mathrm{d}\w\,
\end{equation}
where $\rho_{tot}$ is the total energy density (per area) of the pulse (units, e.g., J/cm$^2$). Therefore, the spectral intensity $\tilde I(\w)$ 
is given by $\tilde I(\w) = 2.05338\cdot10^{-17} |\tilde E_{au}(\w)|^2\ $J\,s/cm$^2$, where $\tilde E_{au}(\w)$ is the output of, e.g., \plotlf{} or \printlf{}.
In atomic units, the spectral intensity is $\tilde I(\w) = 5.45249\,|E_{au}(\w)|^2\ $a.u., where ``a.u.'' here is Hartree $t_{au}/a_0^2$.
Note that the prefactors in $I(t)$ and $\tilde I(\w)$ are the same in atomic units, as it is simply $\epsilon_0 c/2$ in atomic units ($4\pi\epsilon_0=1, c=1/\alpha$),
with $\alpha\approx1/137.036$ the fine structure constant.

Finally, note that the integral in frequency $\w$ runs from $-\infty$ to $\infty$, while \plotlf{} by default only outputs $\w>0$. To perform the full integral, you can use that $\tilde E(-\w)=\tilde E^*(\w)$ since $E(t)$ is real, so that $\tilde I(-\w)=\tilde I(\w)$ and $\int_{-\infty}^\infty \tilde I(\w) \mathrm{d}\w = 2 \int_0^\infty \tilde I(\w) \mathrm{d}\w$. So it would be possible to add a factor $2$ to $\tilde I(\w)$ and have it defined only for $\w>0$.

\end{document}
