\documentclass[letterpaper,aps,pra,10pt,floats,nofootinbib,notitlepage]{revtex4-1}
\usepackage[latin1]{inputenc}
\usepackage[T1]{fontenc}
\usepackage{subfigure}
\usepackage{amsmath,amssymb,mathbbol}

\usepackage{grffile,graphicx,color}
\usepackage{hyperref}

\newcommand{\titlestr}{Fourier transform of chirped pulses}

\hypersetup{colorlinks,%
hypertexnames=true,%
pdfauthor={Johannes Feist},%
pdftitle={\titlestr},%
linkcolor=blue,%
citecolor=blue,%
urlcolor=blue}

\newcommand{\w}{\omega}
\DeclareMathOperator{\erf}{erf}

\begin{document}

\title{\titlestr}
\author{Johannes~Feist}
\email{johannes.feist@gmail.com}
\affiliation{Departamento de F�sica Te�rica de la Materia Condensada, Universidad Aut�noma de Madrid}
\date{\today}

\maketitle

We calculate the Fourier transform of chirped pulses for a few pulse shapes. In the present context, we define
chirped pulses by a linear dependence of the instantaneous frequency on time, i.e.\ $\w(t) = \w_0 + c t$.
Another option would be to define it as a quadratic spectral phase, which is equivalent for Gaussian pulses.
The pulses are assumed to be peaked around $t=0$, as a shift in time is trivially achieved by multiplying the Fourier
transform by $\exp(i \w t_0)$.
All pulses are assumed to be expressed in the form
\begin{align}\label{eq:pulseshape}
E(t) &= f(t) \cos(\phi_0 + \w_0 t + c t^2)\\
     &= \frac12 f(t) e^{i(\phi_0 + \w_0 t + c t^2)} + c.c.\\
     &= \frac12 g(t) e^{i(\phi_0+\w_o t)} + c.c. \,,
\end{align}
where $f(t)$ is the (real) envelope function, while $g(t)=f(t)\exp(ict^2)$ is in general complex.
$\phi_0$ is the carrier-envelope phase. Here and in the following we assume that $f(0) = 1$ and
neglect the trivial scaling with peak field strength. The Fourier transform is then given by
\begin{equation}\label{eq:pulsefourier}
\tilde E(\w) = \frac{e^{i\phi_0}}2 \tilde g(\w-\w_0) + \frac{e^{-i\phi_0}}2 \tilde g^*(\w+\w_0)
\end{equation}
where 
\begin{equation}
\tilde h(\w) = \frac{1}{\sqrt{2\pi}} \int_{-\infty}^{\infty} h(t) e^{-i\w t} \mathrm{d}t
\end{equation}

\section{Gaussian pulse}
A Gaussian pulse is described by $f(t)=\exp\left(-\frac{t^2}{2\sigma^2}\right)$, where $\sigma$ is the
standard deviation. We can then write $g(t) = \exp(-z t^2)$, where $z = \frac{1}{2\sigma^2} - i c$.
The Fourier transform is then
\begin{equation}
\tilde g(\w) = \frac{1}{\sqrt{8z}} e^{-\w^2/4z} \,.
\end{equation}

\section{$\cos^2$ pulse}
This pulse is defined by 
\begin{equation}
f(t) = \begin{cases} \cos\left(\pi t/T\right)^2 & |t| < T/2 \\ 0 & \text{otherwise}\end{cases}
\end{equation}
By expanding the trigonometric function in polynomials, we can write (for $|t|<T/2$):
\begin{equation}
g(t) = \frac{1}{2} e^{i c t^2} + \frac{1}{4} e^{i c t^2- 2 i \pi t/T}+\frac{1}{4} e^{i c t^2+ 2i\pi t/T}
\end{equation}
which can be Fourier-transformed by using
\begin{equation}
\frac{1}{\sqrt{2\pi}}\int_{-T/2}^{T/2} e^{i a t + i b t^2} \mathrm{d}t =
    \frac{\exp(-\frac{ia^2}{4b})}{i\sqrt{8ib}} 
    \left[\erf\left(\frac{a-bT}{\sqrt{4ib}} \right) 
         -\erf\left(\frac{a+bT}{\sqrt{4ib}} \right)\right]
\end{equation}



\end{document}
